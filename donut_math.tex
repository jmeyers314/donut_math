\documentclass{article}
\usepackage[utf8]{inputenc}
\usepackage{geometry}
\usepackage[usenames,dvipsnames,svgnames,table]{xcolor}
%\usepackage{color}
\usepackage{graphicx}
\usepackage{amsmath}
\usepackage{amsfonts}
\usepackage{mathabx}
\usepackage{commath}
\usepackage{bbold}

\geometry{textwidth=6.5in, textheight=9.0in,
    marginparsep=7pt, marginparwidth=.6in}
\setlength{\parindent}{0in}
\setlength{\parskip}{0.08in}

\newcommand{\red}[1]{\textcolor{red}{#1}}
\newcommand{\blue}[1]{\textcolor{blue}{#1}}
\newcommand*{\annot}[1]{\tag*{\footnotesize{\textcolor{gray}{#1}}}}
\newcommand{\like}{\mathcal{L}}
\let\Pr\undefined
\DeclareMathOperator{\R}{R}

\title{Donut Maths}
\author{Josh Meyers}
\date{January 2018}

\begin{document}

We aim to build a model for the HSC field-dependent wavefront that can be trained using out-of-focus
``donut'' images of stars and parameterized in such a way that only a few variables need to be fit
to model arbitrary in-focus exposures.

Our wavefront model for the ith exposure is:

\begin{equation}
    W^i\left(\vec{u}; \vec{\theta}\right) = W_\mathrm{CCD}\left(\R \vec{u}; \R \vec{\theta}\right) + W_\mathrm{tel}\left(\vec{u}; \vec{\theta}\right) + W_\mathrm{pert}^i\left(\vec{u}; \vec{\theta}\right)
\end{equation}

The left hand side $W^i\left(\vec{u}; \vec{\theta}\right)$ is the wavefront for the $i$th exposure
evaluated at pupil position $\vec{u}$ and field position $\vec{\theta}$.  These position vectors
should be understood to be aligned with the altitude and azimuth of an exposure.  Roughly speaking,
$\vec{u}$ corresponds to a point on the primary mirror, independent of exposure, and a
$\vec{\theta}$ indicates a location on the focal plane in the absence of any camera rotation.  The
reality of non-zero camera rotation then implies that a given $\vec{\theta}$ can land anywhere on a
circle in the physical CCD array, depending on the rotator angle.

We model the wavefront in three terms .

\begin{itemize}

  \item $W_\mathrm{CCD}\left(\R \vec{u}; \R \vec{\theta}\right)$  This term is intended to model
  wavefront perturbations that are fixed to the CCD array.  This term may therefore be
  discontinuous across chip boundaries. The 2d rotation operator $\R$ is to account for the camera
  rotator.  When $\R$ acts on $\vec{\theta}$, it is because we are evaluating the wavefront at a
  rotated position compared to where it is defined.  When $\R$ acts on $\vec{u}$, it is because we
  wish to know the wavefront in the alt-az coordinate frame but it is assumed fixed in the CCD
  frame.  This term is static (independent of the exposure index $i$).

  \item $W_\mathrm{tel}\left(\vec{u}; \vec{\theta}\right)$  This term is also static, but aligned
  with the alt-az coordinate system, and roughly corresponds to the wavefront present in the HSC
  design, plus the mean of any additional aberrations resulting from the average gravitational
  loading or permanent misalignments in the optical system.  This term should be continuous over
  the full focal plane.  We also expect that it should be the dominant term, and the other two
  terms are simply perturbations around this one.

  \item $W_\mathrm{pert}^i\left(\vec{u}; \vec{\theta}\right)$  This term is dynamic (depends on
  the exposure index $i$.)  It is intended to capture the effects of flexure, or temperature
  variations, or any other time-dependent perturbations to the wavefront.  This term should also
  be continuous over the full focal plane.

\end{itemize}

In general, we decompose the $\vec{u}$ dependence of the wavefront into Zernike polynomials indexed
by $j$.

\begin{equation}
    W^i\left(\vec{u}; \vec{\theta}\right) = \sum_{j=4}^{j_\mathrm{max}} a_j^i(\vec{\theta}) Z_j(\vec{u})
\end{equation}

The trick is to figure out the $\theta$ dependence of the coefficients.

\section{Principle Component Analysis}

The simplest model may be to assume that $W_\mathrm{CCD} = 0$, and that $W_\mathrm{tel}$ and
$W_\mathrm{pert}^i$ can be decomposed using principle component analysis (PCA) of the available
wavefront constraints from pairs of donut images.  To make this possible, one must first get the
wavefronts for the individual pairs onto the same system to perform the PCA.  One way to do this is
to expand the wavefront from each donut exposure pair in a Zernike basis itself, thus making a
double Zernike basis for the entire field-dependent wavefront:

\begin{equation}
    a_j^i(\vec{\theta}) = \sum_{k=1}^{k_\mathrm{max}} b_{jk}^i Z_k(\vec{\theta})
\end{equation}

Each donut exposure pair has its own set of double Zernike coefficients $b_{jk}^i$.  We would then
associate

\begin{equation}
    W_\mathrm{tel}(\vec{u}, \vec{\theta}) = \sum_{jk} b_{jk}^0 Z_j(\vec{u}) Z_k(\vec{\theta})
\end{equation}

where $b_{jk}^0 = \langle b_{jk}^i \rangle_i$, and

\begin{equation}
    W_\mathrm{pert}^i(\vec{u}, \vec{\theta}) = \sum_{\ell j k} c^i_\ell p^\ell_{jk} Z_j(\vec{u}) Z_k(\vec{\theta})
\end{equation}

where $p^\ell_{jk}$ are the principle components of $b^i_{jk} - \langle b^i_{jk}\rangle_i$, and
$c_\ell^i$ is the projection of the $\ell$th principle component onto the $i$th set of double
Zernike coefficients.  When switching from model building using donut pairs $\{i\}$ to model fitting
of arbitrary in-focus exposures $\{m\}$, one would simply optimize a given loss function over the
$c^m_\ell$s for the $m$th in-focus exposure.

Note that the Zernike orders need not be the same for $W_\mathrm{tel}$ and $W_\mathrm{pert}$. E.g.,
the principle components might only index up to $j^\mathrm{pert}_\mathrm{max}=11$ and
$k^\mathrm{pert}_\mathrm{max}=3$, while the static piece may include terms up to
$j^\mathrm{tel}_\mathrm{max}=21$ and $k^\mathrm{tel}_\mathrm{max}=21$ or something like that.

\section{Full Double Zernike}

We can also think about expanding $W_\mathrm{CCD}$ in a double Zernike basis.

In this case, because there can be discontinuities between the CCDs, we need a separate expansion
for each CCD.  Also, since the CCDs are relatively small regions of the entire focal plane, it
probably makes sense to limit the order of focal plane expansion over each CCD, say $Z_1$, $Z_2$,
and $Z_3$, (effectively a plane).  $W_\mathrm{CCD}$ over the whole focal plane is then the
concatenation in space of the term for each individual CCD:

\begin{equation}
    W^i_\mathrm{CCD}(\vec{u}, \vec{\theta}) = \sum_{n=1}^{n_\mathrm{CCD}} W^i_{\mathrm{CCD},n}(\vec{u}, \vec{\theta}) \mathbb{1}_n(\vec{\theta})
\end{equation}

where the indicator function $\mathbb{1}_n(\vec{\theta})$ is 1 if $\theta$ lands on the $n$th CCD
and 0 otherwise, and $W^i_n(\vec{u}, \vec{\theta})$ is the CCD wavefront for the $n$th CCD.

\begin{equation}
    W^i_\mathrm{CCD,n}(\vec{u}, \vec{\theta}) = \sum_{k=1}^{k^\mathrm{CCD}_\mathrm{max}} \sum_{j=4}^{j^\mathrm{CCD}_\mathrm{max}} d_{njk} Z_j(\vec{u}) Z_k(\vec{\theta})
\end{equation}

The total number of free parameters in this model is then:

\begin{itemize}
    \item $N[W_\mathrm{CCD}] = n_\mathrm{CCD} \times (j^\mathrm{CCD}_\mathrm{max}-3) \times k^\mathrm{CCD}_\mathrm{max}$
    \item $N[W_\mathrm{tel}] = (j^\mathrm{tel}_\mathrm{max}-3) \times k^\mathrm{tel}_\mathrm{max}$
    \item $N[W_\mathrm{pert}] = n_\mathrm{exp} \times (j^\mathrm{pert}_\mathrm{max}-3) \times k^\mathrm{pert}_\mathrm{max}$
    \item $N = N[W_\mathrm{CCD}] + N[W_\mathrm{tel}] + N[W_\mathrm{pert}]$
\end{itemize}

A fully general model for HSC might use the values in Table \ref{tab:1}

\begin{table}[ht]
    \center
    %\caption{Table 1}
    \label{tab:1}
    \begin{tabular}{c|c}
        \hline
        $n_\mathrm{CCD}$ & 104 \\
        $j^\mathrm{CCD}_\mathrm{max}$ & 21 \\
        $k^\mathrm{CCD}_\mathrm{max}$ & 3 \\
        $j^\mathrm{tel}_\mathrm{max}$ & 21 \\
        $k^\mathrm{tel}_\mathrm{max}$ & 21 \\
        $n_{exp}$ & 10 \\
        $j^\mathrm{pert}_\mathrm{max}$ & 21 \\
        $k^\mathrm{pert}_\mathrm{max}$ & 21 \\
        \hline
        $N[W_\mathrm{CCD}]$ & 5616 \\
        $N[W_\mathrm{tel}]$ & 378 \\
        $N[W_\mathrm{pert}]$ & 378 \\
        \hline
        $N$ & 9774
    \end{tabular}
\end{table}

I have roughly 20,000 donut pairs, each measuring 18 Zernike coefficients for a total of $\sim
360000$ data points, so plenty of constraints for 9774 parameters in principle.  Note that this
model is completely linear in the parameters, so linear algebra techniques can be used to find a
solution.

I think it's also the case that the problems for differing pupil Zernikes are almost completely
separable.  For instance, I think the problem for $j=4$ (pupil defocus) doesn't depend on any
coefficients or data except for when $j=4$.  Similarly, $j=5$ and $j=6$ can be solved without
reference to any other $j$s, although the 5 and 6 cannot be further broken apart because these two
Zernikes mix together under a rotation.  Similar statements apply for groups of Zernikes: $\{7,
8\}$, $\{9,10\}$, $\{11\}$, $\{12,13\}$, $\{14,15\}$, $\{16, 17\}$, $\{18,19\}$, $\{20, 21\}$,
$\{22\}$, and so on.

Fitting a full model with 9774 parameters should reveal which parameters are actually important. I'd
say it's very feasible that forcing small parameters to zero could reduce the size of the problem
significantly, say to the values in Table 2.

\begin{table}[ht]
    \center
    %\caption{Table 2}
    \label{tab:2}
    \begin{tabular}{c|c}
        \hline
        $n_\mathrm{CCD}$ & 104 \\
        $j^\mathrm{CCD}_\mathrm{max}$ & 11 \\
        $k^\mathrm{CCD}_\mathrm{max}$ & 1 \\
        $j^\mathrm{tel}_\mathrm{max}$ & 21 \\
        $k^\mathrm{tel}_\mathrm{max}$ & 21 \\
        $n_{exp}$ & 10 \\
        $j^\mathrm{pert}_\mathrm{max}$ & 21 \\
        $k^\mathrm{pert}_\mathrm{max}$ & 3 \\
        \hline
        $N[W_\mathrm{CCD}]$ & 1872 \\
        $N[W_\mathrm{tel}]$ & 378 \\
        $N[W_\mathrm{pert}]$ & 540 \\
        \hline
        $N$ & 2790
    \end{tabular}
\end{table}

It may also become apparent that some parameters vary together.  For instance, the two astigmatism
terms are probably related, as are the two coma terms, two trefoil terms, and so on.  Hopefully this
will be revealed from the large fits above, or could be investigated using raytracing simulations.

When it comes time to fitting parameter values to in-focus images, only the $W_\mathrm{pert}$
parameters would be varied to optimize a given loss function.  Hopefully we'll be able to show that
the number of these parameters that matter is just a few (there are $54$ parameters per exposure in
Table 2, but presumably we could reduce this further as discussed above).

\section{More PCA}

I haven't come up with a completely general PCA that can handle the complications introduced by the
rotator so far.  (Given that the model is linear though, I wouldn't be surprised if there is a way
to still do PCA).  One idea that might be workable though is to first determine $W_\mathrm{CCD}$ and
$W_\mathrm{tel}$, and then do PCA on just the $W^i_\mathrm{pert}$ parameters that remain.

\end{document}
