\documentclass{article}
\usepackage[utf8]{inputenc}
\usepackage{geometry}
\usepackage[usenames,dvipsnames,svgnames,table]{xcolor}
%\usepackage{color}
\usepackage{graphicx}
\usepackage{amsmath}
\usepackage{amsfonts}
\usepackage{mathabx}
\usepackage{commath}
\usepackage{bbold}

\geometry{textwidth=6.5in, textheight=9.0in,
    marginparsep=7pt, marginparwidth=.6in}
\setlength{\parindent}{0in}
\setlength{\parskip}{0.08in}

\newcommand{\red}[1]{\textcolor{red}{#1}}
\newcommand{\blue}[1]{\textcolor{blue}{#1}}
\newcommand*{\annot}[1]{\tag*{\footnotesize{\textcolor{gray}{#1}}}}
\newcommand{\like}{\mathcal{L}}
\let\Pr\undefined
\DeclareMathOperator{\R}{R}

\title{Wavefront model}
\author{Josh Meyers}
\date{May 2018}

\begin{document}

\section{Introduction}

We aim to build a model for the HSC field-dependent optics wavefront that can be
trained using out-of-focus ``donut'' images of stars and then used to fit the
optics contribution of in-focus PSFs.  Developing this model is part of our
ongoing effort to factor the PSF into independent contributions from the
atmosphere, optics and sensors.  Such a factorization has many potential
benefits, the greatest of which is likely the ability to capture all of the
effects of CCD-to-CCD discontinuities into the optical component, allowing the
sensor and particularly the atmospheric component to be smoothly interpolated
across the entire field-of-view.  Additional benefits may also include enabling
a significant part of the PSF variation across the field of view to be captured
using just a few parameters, better characterization of high-frequency PSF
components that are not well constrained empirically using finitely sampled
data, and increased ability to characterize the wavelength dependence of the PSF
by characterizing the disparate wavelength dependencies of the individual
components.

\section{Model}

Our model for the optical part of a single star's PSF is Fourier optics.
Specifically, the optical PSF is described by a pupil-obscuration function and
wavefront function. (See DMTN-064 for more details).  Our primary concern for
this note is the spatial variability of the optical wavefront and therefore the
optical PSF over the field of view.  We break down the wavefront model of the
$i$th exposure into three constituent components as follows:

\begin{equation}
    W^i\left(\vec{u}; \vec{\theta}\right) =
    W_\mathrm{tel}\left(\vec{u}; \vec{\theta}\right) +
    W_\mathrm{visit}^i\left(\vec{u}; \vec{\theta}\right) +
    W_\mathrm{CCD}^\phi\left(\vec{u}; \vec{\theta}\right)
\end{equation}

In the above, $i$ is an index over exposures, $\vec{u}$ is a coordinate in the
entrance pupil, $\vec{\theta}$ is the field angle (or equivalently, a location
on the focal plane).  The coordinate system for $\vec{u}$ and $\vec{\theta}$ are
understood to align with the altitude and azimuth axes of the telescope. Note
that this implies that in the presence of a rotator, they are not fixed with
respect to focal plane or CCDs themselves.

The individual terms of the model are:

\begin{itemize}

  \item $W_\mathrm{tel}\left(\vec{u}; \vec{\theta}\right)$  This term is
  intended to model the wavefront perturbations that are present in the
  otherwise unperturbed optical design of the telescope.  It is independent of
  the exposure index $i$, and continuous over the pupil and field of view.

  \item $W_\mathrm{CCD}^\phi\left(\vec{u}; \vec{\theta}\right)$  This term is
  intended to model wavefront perturbations that are fixed to the CCD array,
  such as may originate from displacements in the heights of each CCD. It is
  continuous in the pupil, but we allow discontinuities between different CCDs
  across the focal plane.  Because this term is constant in focal-plane
  coordinates, but not in our alt-az aligned $\vec{u}$ and $\vec{\theta}$
  coordinates, it also depends on the rotator angle $\phi$.  Aside from the
  implicit dependence of $\phi$ on the exposure index $i$, this term is constant
  in time.

  \item $W_\mathrm{visit}^i\left(\vec{u}; \vec{\theta}\right)$  This dynamic
  term is intended to capture exposure-to-exposure differences (beyond the
  rotator angle), that may be due to flex, temperature variations, or an other
  time-dependent continuous perturbations to the wavefront.  As such, it
  explicitly depends on the exposure index $i$.

\end{itemize}

In DMTN-064, we described how we measured individual wavefronts for a limited
number of pairs of intra- and extra- focal images and specific locations within
the field of view.  For these ``donut'' images, we used a forward model that
decomposed the delivered wavefront into a Zernike polynomial series.  I.e.:

\begin{equation}
    W^i\left(\vec{u}; \vec{\theta}_\ast\right) = \sum_{j=4}^{j_\mathrm{max}} a_j^i(\vec{\theta}_\ast) Z_j(\vec{u})
\end{equation}

It is useful to apply the same decomposition for the $\vec{u}$ dependence of the
model terms.  I.e., for the telescope and visit terms we will write:

\begin{equation}
    W_\mathrm{tel}\left(\vec{u}; \vec{\theta}\right) =
    \sum_j b^\mathrm{tel}_j (\vec{\theta}) Z_j(\vec{u})
\end{equation}

\begin{equation}
    W^i_\mathrm{visit}\left(\vec{u}; \vec{\theta}\right) =
    \sum_j c^i_j (\vec{\theta}) Z_j(\vec{u})
\end{equation}

We will write the CCD term using focal plane coordinates $\vec{u}^\prime$ and
$\vec{\theta}^\prime$ for the moment, since these are the coordinates in which
there is no dependence on the rotator angle $\phi$:

\begin{equation}
    W^\prime_\mathrm{CCD}\left(\vec{u}^\prime; \vec{\theta}^\prime\right) =
    \sum_j d_j (\vec{\theta}^\prime) Z_j(\vec{u}^\prime)
\end{equation}

The trick is to then determine the $\vec{\theta}$ dependence.  For this, we will
again use Zernike polynomials, forming a double zernike basis.  For the
telescope and visit terms this is:

\begin{equation}
    W_\mathrm{tel}\left(\vec{u}; \vec{\theta}\right) =
    \sum_{jk} b^\mathrm{tel}_{jk} Z_k(\vec{\theta}) Z_j(\vec{u})
\end{equation}

\begin{equation}
    W^i_\mathrm{visit}\left(\vec{u}; \vec{\theta}\right) =
    \sum_{jk} c^i_{jk} Z_k(\vec{\theta}) Z_j(\vec{u})
\end{equation}

For the CCD term, we additionally indicate which CCD $n$ the field angle
$\vec{\theta}^\prime$ projects to (using the indicator function
$\mathbb{1}_n(\vec{\theta}^\prime)$):

\begin{equation}
    W^\prime_\mathrm{CCD}\left(\vec{u}^\prime; \vec{\theta}^\prime\right) =
    \sum_{njk} \mathbb{1}_n(\vec{\theta}^\prime) d_{jk} Z_k(\vec{\theta}^\prime) Z_j(\vec{u}^\prime)
\end{equation}

To incorporate the rotator angle into the last equation, we assert

\begin{equation}
    W^\phi_\mathrm{CCD}\left(\vec{u}; \vec{\theta}\right) =
    W^\prime_\mathrm{CCD}\left(R^\phi\vec{u}; R^\phi\vec{\theta}\right)
\end{equation}

where $R^\phi$ is the normal 2D rotation matrix for angle $\phi$.  For a given
donut or star observation at $\vec{\theta}_\ast$ (in alt-az aligned field
angle), this leads to

\begin{equation}
    W^\phi_\mathrm{CCD}\left(\vec{u}; \vec{\theta}_\ast\right) =
    \sum_{njk} \mathbb{1}_n(R^\phi \vec{\theta}_\ast) d_{njk} Z_j(R^\phi \vec{u}) Z_k(R^\phi \vec{\theta}_\ast)
\end{equation}

The $Z_k(R^\phi \vec{\theta}_\ast)$ is simply a real number we can compute.  The
$Z_j(R^\phi \vec{u})$ factor is as yet a function here, however, and we will
find it convenient to further decompose it into a series in $Z_j{\vec{u}}$
(getting rid of the internal $R^\phi$) using the mathematics presented in
(Tatulli (2013) arXiv:1302.7106v1).  Namely

\begin{equation}
    Z_j(R^\phi \vec{u}) = \sum_{j^\prime} M_{j j^\prime}^\phi Z_{j^\prime}(\vec{u})
\end{equation}

where

\begin{equation}
    M_{j j^\prime}^\phi = \delta_{n n^\prime} \times \begin{cases}
    \cos(m \phi), & m = m^\prime \\
    \sin(m \phi), & m = -m^\prime \\
    0 & |m| \ne |m^\prime|
\end{cases}
\end{equation}

where $n$ ($n^\prime$) and $m$ ($m^\prime$) are the radial and azimuthal indices
of the Zernike polynomial with Noll index $j$ ($j^\prime$).  Note that rows or
columns of $M^\phi_{j j^\prime}$ only ever have at most 2 non-zero entries (when
$n = n^\prime$ and $|m| = |m^\prime|$).  With this decomposition, we obtain
for each of $W_\mathrm{tel}$, $W_\mathrm{visit}$, and $W_\mathrm{CCD}$ a series
in pupil Zernikes $Z_j(\vec{u})$.

\section{Donut fits}

To determine the unknown coefficients $b^\mathrm{tel}_{jk}$, $c^i_{jk}$, and
$d_{njk}$ of our model, we minimize the least squares residual over the
wavefronts:

\begin{equation}
\begin{align}
    \mathrm{loss}
    &\propto \sum_{\ast}
    \int \dif{\vec{u}} \left\{\sum_j a_j^\ast Z_j(\vec{u})
    - \sum_{jk} b^\mathrm{tel}_{jk} Z_k(\vec{\theta}_\ast) Z_j(\vec{u})
    - \sum_{ijk} \mathbb{1}_i(\ast) c^i_{jk} Z_k(\vec{\theta}_\ast) Z_j(\vec{u})
    - \sum_{injk} \mathbb{1}_n(R^{\phi_i}) \mathbb{1}_i(\ast) \sum_{j^\prime} d_{n j k} M^{\phi_i}_{j j^\prime} Z_k(\vec{\theta}_\ast) Z_j(\vec{u})
    \right\}^2 \\
    &\propto \sum_{\ast}
    \int \dif{\vec{u}} \left\{\sum_j\left(a_j^\ast
    - \sum_{k} b^\mathrm{tel}_{jk} Z_k(\vec{\theta}_\ast)
    - \sum_{ik} \mathbb{1}_i(\ast) c^i_{jk} Z_k(\vec{\theta}_\ast)
    - \sum_{ink} \mathbb{1}_n(R^{\phi_i}) \mathbb{1}_i(\ast) \sum_{j^\prime} d_{n j k} M^{\phi_i}_{j j^\prime} Z_k(\vec{\theta}_\ast)
    \right) Z_j(\vec{u})\right\}^2 \\
\end{align}
\end{equation}

Expanding the $\left\{\cdot\right\}^2$ above, we can perform the integral over
$\dif{\vec{u}}$ by taking advantage of the orthogonality of Zernike polynomials:

\begin{equation}
    \int Z_j(\vec{u}) Z_{j^\prime}(\vec{u}) \dif \vec{u} = \pi \delta_{j j^\prime}
\end{equation}

Essentially, only the squares of the terms being indexed by $j$ survive, and all
the cross terms vanish.  This leaves us with

\begin{equation}
    \mathrm{loss} \propto \sum_j \sum_{\ast}
    \left\{a_j^\ast
    - \sum_{k} b^\mathrm{tel}_{jk} Z_k(\vec{\theta}_\ast)
    - \sum_{ik} \mathbb{1}_i(\ast) c^i_{jk} Z_k(\vec{\theta}_\ast)
    - \sum_{ink} \mathbb{1}_n(R^{\phi_i}) \mathbb{1}_i(\ast) \sum_{j^\prime} d_{n j k} M^{\phi_i}_{j j^\prime} Z_k(\vec{\theta}_\ast)
    \right\}^2
    \label{eqn:loss}
\end{equation}

which is a sum of positive terms, and can therefore be minimized one value of
$j$ at a time.  And although Eq. \ref{eqn:loss} may look complicated, since the
terms corresponding to the model are linear in each of the unknown variables of
the model ($b^\mathrm{tel}_{jk}$, $c^i_{jk}$, and $d_{njk}$), we can solve it
via standard linear algebra routines.

\section{In-focus fits}
\section{Future}

\end{document}
