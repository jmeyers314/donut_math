\documentclass{article}
\usepackage[utf8]{inputenc}
\usepackage{geometry}
\usepackage[usenames,dvipsnames,svgnames,table]{xcolor}
%\usepackage{color}
\usepackage{graphicx}
\usepackage{amsmath}
\usepackage{amsfonts}
\usepackage{mathabx}
\usepackage{commath}
\usepackage{bbold}

\geometry{textwidth=6.5in, textheight=9.0in,
    marginparsep=7pt, marginparwidth=.6in}
\setlength{\parindent}{0in}
\setlength{\parskip}{0.08in}

\newcommand{\red}[1]{\textcolor{red}{#1}}
\newcommand{\blue}[1]{\textcolor{blue}{#1}}
\newcommand*{\annot}[1]{\tag*{\footnotesize{\textcolor{gray}{#1}}}}
\newcommand{\like}{\mathcal{L}}
\let\Pr\undefined
\DeclareMathOperator{\R}{R}

\title{Wavefront model}
\author{Josh Meyers}
\date{May 2018}

\begin{document}

\section{Introduction}

We aim to build a model for the HSC field-dependent optics wavefront that can be
trained using out-of-focus ``donut'' images of stars and then used to fit the
optics contribution of in-focus PSFs.  Developing this model is part of our
ongoing goal to factor the PSF into independent contributions from the
atmosphere, optics and sensors.  Such a factorization has many potential
benefits, the greatest of which is likely the ability to capture all of the
effects of CCD-to-CCD discontinuities into the optical component, allowing the
sensor  and particularly the atmospheric component to be smoothly interpolated
across the entire field-of-view.  Additional benefits may also include enabling
a significant part of the PSF variation to be captured using just a few
parameters, better characterization of high-frequency PSF components that are
not well constrained empirically using finitely sampled data, and increased
ability to characterize the wavelength dependence of the PSF by characterizing
the disparate wavelength dependencies of the individual components.

\section{Model}

Our model for the optical part of the PSF is Fourier optics.  Specifically, the
optical PSF is described by a pupil-obscuration function and wavefront function.
(See DMTN-064 for more details).  Our primary concern for this note is the
spatial variability of the wavefront both over the pupil plane and over the
focal plane (i.e., the wavefront for a single exposure is a function from
$\mathbb{R}^4$ to $\mathbb{R}$).  We break down the wavefront model of the $i$th
exposure into three constituent components as follows:

\begin{equation}
    W^i\left(\vec{u}; \vec{\theta}\right) =
    W_\mathrm{tel}\left(\vec{u}; \vec{\theta}\right) +
    W_\mathrm{visit}^i\left(\vec{u}; \vec{\theta}\right) +
    W_\mathrm{CCD}\left(\R_\phi \vec{u}; \R_\phi \vec{\theta}\right)
\end{equation}

In the above, $i$ is an index over exposures, $\vec{u}$ is a coordinate in the
entrance pupil, $\vec{\theta}$ is the field angle (or equivalently, a location
on the focal plane).  The coordinate system for $\vec{u}$ and $\vec{\theta}$ are
understood to align with the altitude and azimuth axes of the telescope.
This means that in the presence of a rotator, they are not fixed with respect to
focal plane or CCDs themselves.

The individual terms of the model are then:

\begin{itemize}

  \item $W_\mathrm{tel}\left(\vec{u}; \vec{\theta}\right)$  This term is
  intended to model the wavefront perturbations that are present in the
  otherwise unperturbed optical design of the telescope.  It is independent of
  the exposure index $i$, and continuous over the pupil and field of view.

  \item $W_\mathrm{CCD}\left(\R_\phi \vec{u}; \R_\phi \vec{\theta}\right)$  This
  term is intended to model wavefront perturbations that are fixed to the CCD
  array, such as may originate from displacements in the height of each CCD.  It
  is continuous in the pupil, by we allow discontinuities between different CCDs
  across the focal plane.  Because this term is constant in focal-plane
  coordinates, but not in our alt-az aligned $\vec{u}$ and $\vec{\theta}$
  coordinates, we are required to rotate from one system to the other.  This is
  accomplished by the rotation operators $\R_\phi$, where $\phi$ indicates the
  angle of rotation.  This term also does not depend on the exposure index $i$.

  \item $W_\mathrm{visit}^i\left(\vec{u}; \vec{\theta}\right)$  This term is
  dynamic (depends on the exposure index $i$.)  It is intended to capture the
  effects of flexure, or temperature variations, or any other time-dependent
  continuous perturbations to the wavefront.

\end{itemize}

Need to break down individual terms into double zernike basis.

In DMTN-064, we described how we measured individual wavefronts for a limited
number of pairs of intra- and extra- focal images and specific locations within
the field of view.  For these "donut" images, we used a forward model that
decomposed the delivered wavefront into a Zernike polynomial series.  I.e.:

\begin{equation}
    W^i\left(\vec{u}; \vec{\theta_\ast}\right) = \sum_{j=4}^{j_\mathrm{max}} a_j^i(\vec{\theta_\ast}) Z_j(\vec{u})
\end{equation}



\section{Donut fits}
\section{In-focus fits}
\section{Future}

\end{document}
